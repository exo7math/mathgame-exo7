\newcounter{num}
\newcommand{\tictactoe}[2][1]
{
	\begin{tikzpicture}[line width = #1*2pt, ,scale=#1, transform shape]
		\def\r{3mm}
		\tikzset{
			circ/.pic={\draw circle (\r);},
			cross/.pic={\draw (-\r,-\r) -- (\r,\r) (-\r,\r) -- (\r,-\r);},
			opto/.pic={\draw[red, opacity=0.7] circle (\r);},
			opt/.pic={\draw[red, opacity=0.7] (-\r,-\r) -- (\r,\r) (-\r,\r) -- (\r,-\r);}
		}		
		% The grid
		\foreach \i in {1,2} \draw (\i,0) -- (\i,3) (0,\i) -- (3,\i);		
		% Numbering the cells
		\setcounter{num}{0}
		\foreach \y in {0,...,2}
		\foreach \x in {0,...,2}
		{
			\coordinate (\thenum) at (\x+0.5,2-\y+0.5);
			%\node[opacity=0.5] at (\thenum) {\sffamily\thenum}; % Uncomment to see numbers in the cells
			\addtocounter{num}{1}
		}		
		\def\X{X} \def\x{x} \def\O{O} \def\o{o} \def\n{n}		
		\foreach \l [count = \i from 0] in {#2}
		{
			\if\l\X \path (\i) pic{cross};
			\else
			\if\l\O \path (\i) pic{circ};
			\else
			\if\l\x \path (\i) pic{opt};
			\else
            \if\l\o \path (\i) pic{opto};			
			\else
			\if\l\n \node[opacity=0.5] at (\i) {\sffamily\i};
			\fi
			\fi
			\fi
			\fi
			\fi
		}
	\end{tikzpicture}
}