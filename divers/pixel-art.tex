% Pixel art pour les dessin à chaque début de partie du livre	
\newcommand{\pixelpart}[3]{
	\ifthenelse{#3=0} {\fill[white] (#1-0.48,#2-0.48) rectangle ++(0.96,0.96);} {}
	\ifthenelse{#3=1} {\fill[black] (#1-0.48,#2-0.48) rectangle ++(0.96,0.96);} {}
	\ifthenelse{#3=2} {\fill[yellow] (#1-0.48,#2-0.48) rectangle ++(0.96,0.96);} {}
	\ifthenelse{#3=3} {\fill[green] (#1-0.48,#2-0.48) rectangle ++(0.96,0.96);} {}
	\ifthenelse{#3=4} {\fill[violet!70] (#1-0.48,#2-0.48) rectangle ++(0.96,0.96);} {}
	\ifthenelse{#3=5} {\fill[blue] (#1-0.48,#2-0.48) rectangle ++(0.96,0.96);} {}
	\ifthenelse{#3=6} {\fill[cyan] (#1-0.48,#2-0.48) rectangle ++(0.96,0.96);} {}
}
	
\newcommand\mypixelart[1]{	
	\ifnum #1=1
	% Part 1 : space Invader
	\def\xmax{10}
	\def\ymax{8}
	\def\scalepixel{4/\ymax}  % 
	\def\PixelArt{{ 
			{0,0,1,0,0,0,0,1,0,0},
			{1,0,0,1,0,0,1,0,0,1},
			{1,0,1,1,1,1,1,1,0,1},
			{1,1,1,0,1,1,0,1,1,1},
			{1,1,1,1,1,1,1,1,1,1},
			{0,1,1,1,1,1,1,1,1,0},
			{0,0,1,0,0,0,0,1,0,0},
			{0,1,0,0,0,0,0,0,1,0}
	}} 
	\fi
	
	\ifnum #1=2
	
	% Part 2 Pacman
	 \def\xmax{14}
	 \def\ymax{15}
	 \def\scalepixel{4/\ymax}	
	 \def\PixelArt{{ 
		 {0,0,0,0,0,0,1,1,1,0,0,0,0,0},
		 {0,0,0,0,1,1,2,2,2,1,1,0,0,0},
		 {0,0,0,1,2,2,2,2,2,2,2,1,0,0},
		 {0,0,1,2,2,2,2,2,2,2,2,2,1,0},
		 {0,1,2,2,2,2,2,1,2,2,2,2,2,1},
		 {0,1,2,2,2,2,2,2,2,2,2,1,1,0},
		 {1,2,2,2,2,2,2,2,1,1,1,0,0,0},
		 {1,2,2,2,2,2,1,1,0,0,0,0,0,0},
		 {1,2,2,2,2,2,2,2,1,1,1,0,0,0},
		 {0,1,2,2,2,2,2,2,2,2,2,1,1,0},
		 {0,1,2,2,2,2,2,2,2,2,2,2,2,1},
		 {0,0,1,2,2,2,2,2,2,2,2,2,1,0},
		 {0,0,0,1,2,2,2,2,2,2,2,1,0,0},
		 {0,0,0,0,1,1,2,2,2,1,1,0,0,0},
		 {0,0,0,0,0,0,1,1,1,0,0,0,0,0},
		 }} 		
	\fi	
	
	\ifnum #1=3	
	
	% Part 3 : Tetris
	\def\xmax{8}
	\def\ymax{3}
	\def\scalepixel{4/\ymax}  % 	
	\def\PixelArt{{ 
		{0,2,2,3,0,4,0,5},
		{0,2,2,3,3,4,4,5},
		{6,6,6,6,3,4,5,5}
}} 	
	
	\fi	
	
	\ifnum #1=4

% Part 4 : Echec
	\def\xmax{12}
	\def\ymax{17}
	\def\scalepixel{4/\ymax}  % 
	\def\PixelArt{{ 
		{0,0,0,0,0,0,1,0,0,1,0,0},
		{0,0,0,0,0,1,1,1,1,1,0,0},
		{0,0,0,0,1,1,1,1,1,1,1,0},
		{0,0,0,1,1,1,0,1,1,1,1,1},
		{0,0,1,1,1,1,1,1,1,1,1,0},
		{0,1,1,1,1,1,1,1,1,1,1,1},
		{1,0,1,1,1,1,1,1,1,1,1,1},
		{1,1,1,1,1,1,1,1,1,1,1,0},
		{0,1,1,0,0,1,1,1,1,1,1,1},
		{0,0,0,0,1,1,1,1,1,1,0,0},
		{0,0,0,1,1,1,1,1,1,1,0,0},
		{0,0,0,1,1,1,1,1,1,1,0,0},
		{0,0,0,1,1,1,1,1,1,1,1,0},
		{0,0,0,0,1,1,1,1,1,1,1,0},
		{0,0,0,0,0,0,0,0,0,0,0,0},
		{0,0,1,1,1,1,1,1,1,1,1,1},
		{0,0,1,1,1,1,1,1,1,1,1,1}
}} 
    \fi		
		
	% Plot pixels
	\begin{center}	
		\begin{tikzpicture}[scale = \scalepixel]
			
			
			% Draw pixel art	
			\foreach \y in {1,...,\ymax}{
				\foreach \x in {1,...,\xmax}
				{
					\pgfmathtruncatemacro{\coul} { int(\PixelArt[\y-1][\x-1]) }
					\pixelpart{\x}{\ymax-\y}{\coul}
				}
			}
		\end{tikzpicture}
	\end{center}		
	
} % Fin 