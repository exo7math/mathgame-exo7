
\clearemptydoublepage
\pagestyle{empty}\thispagestyle{empty}

\vspace*{\fill}

\section*{Notes et bibliographie}

%Retrouvez ce cours en vidéos : 
%\mycenterline{
%    \href{https://www.youtube.com/}
%    {Chaine \og{}Mathgame\fg{} sur Youtube}
%}


\textbf{Mathématiques}

Vous trouverez un cours complet d'algèbre et d'analyse avec exercices et vidéos ici : \mycenterline{\href{http://exo7.emath.fr/}{exo7.emath.fr}}

\smallskip

Toutes les ressources ci-dessous sont en anglais.


\smallskip


\textbf{Mathématiques et jeux vidéo}

Voici quelques chaînes proposant des vidéos intéressantes :
\begin{itemize}	
  \item \href{https://www.youtube.com/@g5min}{www.youtube.com/@g5min}

  \item \href{https://www.youtube.com/@codingmath}{www.youtube.com/@codingmath}

  \item \href{https://www.youtube.com/@JorgeVinoRodriguez}{www.youtube.com/@JorgeVinoRodriguez}

  \item \href{https://www.youtube.com/@Acegikmo}{www.youtube.com/@Acegikmo}
\end{itemize}

\smallskip

\textbf{Graphisme}

\begin{itemize}
	\item \href{https://www.scratchapixel.com/}{www.scratchapixel.com} : un site très clair qui explique les principes fondamentaux pour générer l'image d'une scène 3D.
	
    \item \emph{Mathematics for	3D Game Programming	and Computer Graphics} de Lengyel (Course Technology) : tout est dans le titre !	
	
	\item \emph{Computational Geometry} de Berg, Cheong, van Kreveld et Overmars (Springer) et 
	\emph{Discrete and Computational Geometry} de Devadoss et O'Rourke (Princeton University Press) :
	deux livres de géométrie algorithmique à propos des graphes, des polygones, des triangulations.

\end{itemize}

\smallskip

\textbf{Logiciels}


\begin{itemize}
	\item \emph{Pygame} est idéal pour développer des petits jeux 2D avec \Python{} : \href{https://www.pygame.org/}{www.pygame.org}	
	\item \emph{Blender} permet de créer des images 3D ainsi que des petites animations : \href{https://www.blender.org/}{www.blender.org}
\end{itemize}



\section*{Remerciements}

Je remercie Stéphanie Bodin et Michel Bodin pour leurs relectures. Merci à Kroum Tzanez pour certaines figures.






\bigskip

\begin{center}
Vous pouvez récupérer l'intégralité des codes \Python{} ainsi que tous les fichiers sources sur la page \emph{GitHub} d'Exo7 :
\href{https://github.com/exo7math/mathgame-exo7}{\og{}GitHub : Mathgame\fg{}}.


\end{center}


\vspace*{\fill}

\bigskip 

\begin{center}
\LogoExoSept{3}
\end{center}



\begin{center}
Ce livre est diffusé sous la licence \emph{Creative Commons -- BY-NC-SA -- 4.0 FR}.

Sur le site Exo7 vous pouvez télécharger gratuitement le livre en couleurs.
\end{center}




\pagenumbering{gobble} % remove page numbering 
\printindex
\pagenumbering{arabic}

